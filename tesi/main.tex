%%%%%%%%%%%%%%%%%%%%%%%%%%%%%%%%%%%%%%%%%%%%%%%%%%%%
%
% Copyright 2022 by Isac Pasianotto
%
% This file may be distributed and/or modified
%
% 1. under the LaTeX Project Public License and/or
% 2. under the GNU Public License.
%
%%%%%%%%%%%%%%%%%%%%%%%%%%%%%%%%%%%%%%%%%%%%%%%%%%%%%%%%%%%


%%%%%%%%%%%%%%%%%%%%%%%%%%%%%
%%%%   NOTE IMPORTANTI   %%%%
%%%%%%%%%%%%%%%%%%%%%%%%%%%%%%

%	Alcuni aspetti da sapere per poter usare il presente template: 
%	
%	1. Ricordarsi di editare le proprietà del pdf (autore, titolo) nel file otherResources/allPackages.sty
%	2. La compilazione, a causa dell'uso del paccheetto \usepackage{frontespizio} deve avvenire in due fasi: 
%			a.   pdflatex/latex/xelatex/...   main.tex
%			b.   pdflatex/latex/xelatex/...   main-frn.tex
%			c.   pdflatex/latex/xelatex/...   main.tex        (per unire il frontespizio al resto del documento)
%	3. Il file "main-frn.tex" è generato in automatico e salvo casi molto eccezzionali, non andrebbe mai modificato. Lavorare su 
%      "otherResources/frontespizio.tex" invece. 	
%	4.  Solamente quando il frontespizio ha raggiunto il suo aspetto definitivo, si può evitare la procedura 
%		descritta al punto 2, aggiungere ai parametri del pacchetto frontespizio caricato in "otherResources/allPackages" il 
%		flag 'nowrite'.
%	5.  Per sistemi unix-like lo script runAll, richiamabile col comando ./runAll.sh dovrebbe fare tutto in automatico. Altrimenti 
%       fare attenzione a come si è impostato l'editor di LaTeX in uso. 
%       Per es. controllare che sia selezionato BibTex anzichè Biber etc etc


\documentclass[openany]{report}	%cambio font-size con: [12pt, openany]{book}

% Tutti i pacchetti sono caricati in un file esterno: 
\usepackage{otherResources/allPackages}

%%%%%%%%%%%%%%%%%%%%%%%%%%%%%%%%%%%%%
%% CARICAMENTO VARI CAPITOLI TESI %%
%%%%%%%%%%%%%%%%%%%%%%%%%%%%%%%%%%%%%

\begin{document}
	% Frontespizio
	\begin{frontespizio}
	\Istituzione{UNIVERSIT\`A DEGLI STUDI TRIESTE}
	\Dipartimento{Scienze Economiche, Aziendali, Matematiche e Statistiche}
	\Logo[3.5cm]{otherResources/logo-units}
	\Corso[Laurea]{Statistica e informatica per l'azienda, la finanza e l'assicurazione}
	\Annoaccademico{2021--2022}
	\Titoletto{Tesi di Laurea}
	\Titolo{TITOLO DELLA TESI}
	\Sottotitolo{Sottotitolo della tesi se presente}
	\Candidato{Jean-luc Picard}
	\Relatore{Chiar.mo Prof.\\ James T. Kirk}
	\Margini{2.5cm}{2.5cm}{2.5cm}{2.5cm}
	%\Filigrana[height=4cm,before=2.6,after=8.8]{logo-units}
	\Punteggiatura{}
	\NCandidato{\qquad Laureando \qquad} %lo centro " a mano"
	\NRelatore{\qquad Relatore \qquad}{\qquad Relatori \qquad}
	\Rientro{2.5cm}
\end{frontespizio}	
	\pagenumbering{roman}
	% Dedica e/o frase in prima di copertina
	 \myemptypage
\begin{titlepage}
	
	\nonumber
	\null \vspace {\stretch{1}}
	\begin{flushright}
			%\begin{verse}
		\textit{I have been \\ 
			- and always shall be - \\
			your friend.\\
			%\ \\ %stampo un carattere bianco per evitare errori
			Live long and prosper} \\[5mm]
		%	\end{verse}
		
		\ \\
		Mr. Spock to Captain Kirk 
	\end{flushright}
	\vspace{\stretch{2}}\null
	
\end{titlepage}

	\myemptypage 
	% Indice: 
	\newpage
	\tableofcontents
	% Introduzione
	\pagenumbering{arabic}
	\chapter*{Introduzione} 

Introduzione della tesi.   Non appare nella numerazione dei capitoli.
	\thispagestyle{empty} % Tolgo il titolo sbagliato in alto a dx su l'introduzione (compariva "indice")
	% Capitolo 1: Titolo capitolo 1
	\chapter{Capitolo 1} 

Magari sforzarsi di scrivere qualcosa in più che semplicemente "capitolo 1".

Per citare, usare il comando \textit{cite}: per es \cite{pauli:ModelloLineare}
	% Capitolo 2: Titolo capitolo 2
	\chapter{Capitolo 2}

Capitolo 2.
	% Capitolo 3: Titolo capitolo 3
	\chapter{Capitolo 3}

Capititolo 3... Andare avanti ad aggiungere capitoli secondo questo schema
	% Appendici
    	\begin{appendices}
\chapter{Titolo della appendice}
Questa è una sezione dopo tutti i capitoli.
\end{appendices}

	% ...	
	% Bibliogragia
	% Uso questo file per inserire tutte le fonti bibliografiche senza nessuna citazione diretta nel corpo della tesi 

\nocite{Shannon:introduzione}
	%Valutare se stamparla tutta in un colpo oppure utilizzare le keyword e stamparla divisa in sezioni
	\printbibliography[title={Riferimenti bibliografici e sitografia}]
	% Ringraziamenti 
    \myemptypage\newpage
\myemptypage\newpage % uno o due in base a come finisce la bibliografia
\hyphenpenalty=10000
\begin{titlepage}	
\nonumber
\null \vspace {\stretch{0.5}}
\begin{flushright}
	\textit{\large Ringraziamenti}: \\
	\ \\
    \begin{minipage}{0.8\textwidth}
    
    		A me personalmente piaceva questo tipo di layout per i ringraziamenti. Essendo una cosa molto soggettiva, e non essendo previste delle regole specifiche, 
    		ognuno faccia come meglio ritiene. 
	\end{minipage}
\end{flushright}

\vspace{\stretch{2}}\null
\end{titlepage}

\end{document}
	
